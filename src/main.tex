% --------------------------------------------------------------------------------------------
% ---------- 日本応用数理学会年会予稿集原稿テンプレート ----------
% --------------------------------------------------------------------------------------------
%
% ---------- 変更しないでください(ここから) ----------
\documentclass[11pt,a4j]{jarticle}
\setlength{\oddsidemargin}{0cm}
\setlength{\topmargin}{-.5cm}
\setlength{\headheight}{0cm}
\setlength{\headsep}{0cm}
\setlength{\footskip}{0cm}
\setlength{\textwidth}{16cm}
\setlength{\textheight}{24.7cm}
\setlength{\abovecaptionskip}{0cm}
\pagestyle{empty}
\makeatletter
\renewcommand{\baselinestretch}{1.3}\selectfont
\def\Title#1{{\Large\bf#1}\\[6pt]}
\def\Author#1{{\normalsize\hspace*{2zw}#1}\\[-4pt]}
\def\Affiliation#1{{\normalsize\hspace*{2zw}#1}\\[-5pt]}
\def\Email#1{\hspace*{2zw}e-mail : #1\\[0pt]}
\renewcommand{\section}{\@startsection{section}{1}{\z@}%
{2ex}{1ex}{\reset@font\large\bfseries}}%
\renewcommand{\thesection}{\@arabic\c@section}
\def\@listi{\topsep=.3\baselineskip \parsep=.2ex \partopsep=0ex%
\itemsep=0ex \leftmargin=4ex \rightmargin=2ex}
\let\@listI\@listi
\@listi\def\@listii{\parsep=.2ex \partopsep=0pt \itemsep=0ex%
\leftmargin=4ex \rightmargin=0ex}
\let\@listiii\@listii
\let\@listiv\@listii
\let\@listv\@listii
\let\@listvi\@listii
\long\def\@makecaption#1#2{\footnotesize\sbox\@tempboxa{#1. #2}
\ifdim\wd\@tempboxa >\hsize #1. #2\par
\else \global\@minipagefalse
\hb@xt@\hsize{\hfil\box\@tempboxa\hfil}
\fi}
\makeatother
% ---------- 変更しないでください(ここまで) ----------
%
% ---------- お好みで変更および追加してください(ここから) ----------
\usepackage{graphicx}
\usepackage{amsmath}
\usepackage{url}
\newtheorem{thm}{定理}
\newtheorem{df}[thm]{定義}
\newtheorem{lem}[thm]{補助定理}
\newtheorem{prop}[thm]{補題}
\def\proof{{\bf 証明}\hspace*{1zw}}
\def\thanks{~\\[.5\baselineskip]{\bf 謝辞}\hspace*{1zw}}
\def\labelenumi{\theenumi)}
% ---------- お好みで変更および追加してください(ここまで) ----------
%
\begin{document}
\twocolumn[
%
% ---------- タイトル,著者名,所属,e-mailアドレスを記入してください(ここから) ----------
\Title{On the security of QUIC}
%%% 著者,所属が複数行になる場合は \author{},\affiliation{}を追加してください
\Author{Masaya Iseki$^{1}$, Eiichiro Fujisaki$^{2}$}
\Affiliation{$^{1}$Tokyo Institute of Technology ,$^{2}$NTT Secure platform}
\Email{iseki.m.aa@m.titech.ac.jp, fujisaki.eiichiro@lab.ntt.co.jp}
% ---------- タイトル,著者名,所属,e-mailアドレスを記入してください(ここまで) ----------
%
]
% ---------- 変更しないでください(ここから) ----------
\renewcommand{\baselinestretch}{0.95}\selectfont
% ---------- 変更しないでください(ここまで) ----------
%
% ---------- 本文(ここから)--------------------------------------------------------------------------------------------
%
% ---------------------------------------
\section{Introduction}
% ---------------------------------------

Quick UDP Internet Connections (QUIC for short)
is a new transport layer network protocol recently proposed by Google~\cite{QUIC},
which is experimentally implemented in Google Chrome.
The main purpose of developing QUIC is to provide an alternative equivalence of TLS wrapping TCP,
with much reduced latency and better SPDY support.
Transport Layer Security (TLS) starts with a three-move TCP handshake
before initiating the TLS Handshake Protocol.
In contrast, QUIC uses UDP and starts with its own handshake,
which reduces the total number of interactions.
The cryptographic core of QUIC is specified in the QUIC crypto protocol~\cite{QUIC:Crypto},
which consists of a handshake protocol and a record layer protocol,
as does TLS.
Similarly to TLS, QUIC has two types of handshake connections.
One is called a full handshake --
a handshake ``from scratch" between a client and a server.
The other is called a abbreviate handshake -- a handshake which occurs when a client and a server have once established a full handshake session
and want to establish a new session between them in an abbreviate way.
Unlike TLS,
QUIC only supports the elliptic-curve Diffie-Hellman key-exchange (ECDHE) cipher-suite and server authentication.
%
One of the good features of QUIC is that it can establish a abbreviate handshake session
with $0$-RTT connectivity overhead.
Namely, in the QUIC abbreviate handshake, a client can
send encrypted data to a server, concurrently with establishing a new session.
We provide the abstract model of the full and abbreviate handshake and of QUIC.

% ---------------------------------------
\section{Prior Security Analyses and Some Security Concern}
% ---------------------------------------

To the best of our knowledge, there are only two security analyses on QUIC~\cite{FG14:QUIC,LJBN15:QUIC}.
Both papers define new security models and show that QUIC is secure in that model.
In~\cite{LJBN15:QUIC}, they formalized a secure authenticated key-exchange
as an extension of the ACCE model~\cite{JKSS12:ACCE} and analysed
the security of QUIC.
They prove the security of QUIC as it is and found a new attack.


%=====================================
\subsection{Related Work} \label{sec:Related Work}
%=====================================
There are a huge body of works on authenticated key exchange protocols
(See~\cite{CK01:AKE} for survey).
An important stream of research dates back to Bellare and Rogaway~\cite{BR93:AKE}.
% followed by~\cite{DB96,Blei98,JMDP00,JB02,EK09,KK05:TLS,KCRE08,SMOAJ08,KTT11,Kraw01}.
However, as mentioned above,
the QUIC full handshake protocol does not satisfy key-indistinguishability as in the Bellare-Rogaway like model, because a server sends a ciphertext (using authenitcated encryption) in the full handshake protocol, as does TLS.
TLS Handshake Protocol is recently analysed in various security models, e.g., ~\cite{JKSS12:ACCE,KPW13:SACCE,FS13:ACCE,GKS13:RACCE,BDKSS14:SSH,BFKPSB14:TLS}.
Still,
the security model for analysing a server-only authenticated connection of TLS,
i.e., Server-Only Authenticated and Confidential Channel Establishment
(SACCE)~\cite{KPW13:SACCE}, does not capture our security issues.
This is because, besides not treating the resumption originally,
the SACCE security model only concerns, which is more essential,
\textit{server} authentication and a \textit{client's} message confidentiality.
However, these issues appear in (some kind of) \textit{client} authentication
and \textit{server} confidentiality.

%--------------------------------------------------------
\subsection{Our Results} \label{sec:proposal}
%--------------------------------------------------------

To treat the security issues above,
we introduce a new security model, what we call \textit{Resumable} SACCE (RSACCE) security,
where we consider a server's message confidentiality, as well as a client's message confidentiality,
where an adversary is allowed to send an encryption query to a \textit{server}
(to break a server's message confidentiality)
both in the full handshake session and its successor resumption sessions,
as far as the server establishes the initial full handshake session
with a \textit{honest} client.
We also provide a stronger model, called strong RSACCE security,
where we ensure that a sever can establish a resumption session only with the \textit{same} client as
initially connected in the full handshake session.
We require in that model forward secrecy among all independent sessions.
For resumption to be effective, we compromise but still require
some level of forward secrecy among related resumption sessions.


%
We analyse QUIC as it is, and prove that QUIC meets RSACCE security, but it does not meet
the strong RSACCE one.
We also analyse an optional version of QUIC with CETV, QUIC with an optional client encrypted tag value
(CETV) mechanism, and show that it meets strong RSACCE security.

% -------
\thanks
% -------
%
\dotfill \\ \dotfill

\bibliographystyle{junsrt}
\bibliography{mybib,confCryp,confComp}

\end{document}
